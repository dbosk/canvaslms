\chapter{Testing strategy and shared fixtures}% ===> this file was generated automatically by noweave --- better not edit it
\chapterprecis{%
  Written by Dan-Claude under supervision of Daniel Bosk.
}

Testing a command-line interface to an external API presents unique challenges.
We cannot make real API calls during tests---they would be slow, unreliable, and
might modify production data.
Instead, we must create a comprehensive mocking strategy that simulates the
Canvas API's behavior while allowing us to test our code's logic in isolation.

This chapter explains our testing philosophy and provides shared fixtures that
all tests can reuse.
By understanding these foundations first, you'll better appreciate the test
examples embedded in subsequent chapters.

\section{The testing challenge: mocking an external API}

The {\Tt{}canvaslms\nwendquote} tool interacts extensively with the Canvas LMS API through
the {\Tt{}canvasapi\nwendquote} Python library.
Every command---listing courses, filtering assignments, fetching
submissions---makes API calls that return Canvas objects.

\subsection{What we need to test}

We want to verify that our code correctly:
\begin{enumerate}
\item Filters and processes Canvas objects based on command-line arguments
\item Handles edge cases like empty results, regex matching, and missing data
\item Raises appropriate errors (like {\Tt{}EmptyListError\nwendquote}) when expected
\item Formats output correctly for Unix pipelines
\end{enumerate}

What we \emph{don't} need to test:
\begin{enumerate}
\item Whether the Canvas API returns correct data (that's Canvas's job)
\item Whether {\Tt{}canvasapi\nwendquote} library works (that's their test suite's job)
\item Network reliability, authentication, rate limiting (external concerns)
\end{enumerate}

\subsection{Why we mock at the API boundary}

We mock Canvas API objects rather than our own functions because:
\begin{description}
\item[Isolation:] Tests verify our logic without depending on external services
\item[Speed:] Mock objects are instantaneous; API calls take seconds
\item[Repeatability:] Tests produce identical results every run
\item[Control:] We can simulate edge cases that are hard to create in production
\end{description}

\subsection{Variation theory perspective: what varies, what stays constant}

When designing our mocks, we apply variation theory:
\begin{description}
\item[Invariant (stays constant):] The Canvas API interface---methods like
{\Tt{}get{\_}courses()\nwendquote}, {\Tt{}get{\_}assignments()\nwendquote}, object attributes like {\Tt{}id\nwendquote},
{\Tt{}name\nwendquote}, {\Tt{}due{\_}at\nwendquote}
\item[Variant (what we change):] The data returned---course names, assignment
counts, submission states, timestamps
\end{description}

By keeping the interface constant while varying the data, our tests focus on
how our code handles different scenarios rather than implementation details.

\section{Pytest and fixtures}

We use {\Tt{}pytest\nwendquote} as our testing framework.
Pytest provides \emph{fixtures}---reusable test components that can be injected
into test functions.

\subsection{What is a fixture?}

A fixture is a function that creates and returns a test resource.
For example:
\begin{verbatim}
@pytest.fixture
def mock_course():
    """Returns a fake Canvas Course object"""
    course = MagicMock()
    course.id = 12345
    course.name = "Introduction to Testing"
    return course

def test_course_filtering(mock_course):
    # pytest automatically calls mock_course() and passes result
    assert mock_course.name == "Introduction to Testing"
\end{verbatim}

Fixtures promote \emph{separation of concerns}:
\begin{itemize}
\item Test setup lives in fixtures
\item Test logic lives in test functions
\item Tests stay focused and readable
\end{itemize}

\subsection{Our fixture strategy}

We provide fixtures at multiple levels of abstraction:
\begin{description}
\item[Canvas instance:] {\Tt{}mock{\_}canvas()\nwendquote}---the root API client
\item[Course objects:] {\Tt{}mock{\_}course()\nwendquote}, {\Tt{}mock{\_}courses()\nwendquote}---single and
multiple courses
\item[Assignment objects:] {\Tt{}mock{\_}assignment()\nwendquote},
{\Tt{}mock{\_}assignment{\_}group()\nwendquote}---assignments and groups
\item[User objects:] {\Tt{}mock{\_}user()\nwendquote}, {\Tt{}mock{\_}group()\nwendquote}---users and groups
\item[Submission objects:] {\Tt{}mock{\_}submission()\nwendquote}---student submissions
\end{description}

Each fixture represents a Canvas API object with realistic attributes.
Tests can customize these attributes as needed.

\section{Implementation: the conftest.py fixture file}

Pytest automatically loads fixtures from {\Tt{}conftest.py\nwendquote} files.
We'll create fixtures for each major Canvas API object type, using Python's
{\Tt{}unittest.mock.MagicMock\nwendquote} to create flexible mock objects.

\nwfilename{conftest.nw}\nwbegincode{1}\sublabel{NW2WrsP2-3ri0xc-1}\nwmargintag{{\nwtagstyle{}\subpageref{NW2WrsP2-3ri0xc-1}}}\moddef{conftest.py~{\nwtagstyle{}\subpageref{NW2WrsP2-3ri0xc-1}}}\endmoddef\nwstartdeflinemarkup\nwenddeflinemarkup
"""
Shared pytest fixtures for testing canvaslms.

This file provides mock Canvas API objects that tests can use
instead of making real API calls. Each fixture returns a MagicMock
configured to behave like the corresponding Canvas API object.
"""
import pytest
from unittest.mock import MagicMock, Mock
from datetime import datetime, timezone
import arrow

\LA{}mock canvas api fixture~{\nwtagstyle{}\subpageref{NW2WrsP2-2nmrdH-1}}\RA{}
\LA{}mock course fixtures~{\nwtagstyle{}\subpageref{NW2WrsP2-4SoGVR-1}}\RA{}
\LA{}mock assignment fixtures~{\nwtagstyle{}\subpageref{NW2WrsP2-3VuERV-1}}\RA{}
\LA{}mock user and group fixtures~{\nwtagstyle{}\subpageref{NW2WrsP2-htc1N-1}}\RA{}
\LA{}mock submission fixtures~{\nwtagstyle{}\subpageref{NW2WrsP2-qDrc3-1}}\RA{}
\LA{}mock configuration fixture~{\nwtagstyle{}\subpageref{NW2WrsP2-2fL3Kj-1}}\RA{}
\nwnotused{conftest.py}\nwendcode{}\nwbegindocs{2}\nwdocspar

\subsection{The mock Canvas API instance}

Every command receives a {\Tt{}canvas\nwendquote} parameter---an instance of the Canvas API
client.
We mock this as the root of our API object hierarchy.

\nwenddocs{}\nwbegincode{3}\sublabel{NW2WrsP2-2nmrdH-1}\nwmargintag{{\nwtagstyle{}\subpageref{NW2WrsP2-2nmrdH-1}}}\moddef{mock canvas api fixture~{\nwtagstyle{}\subpageref{NW2WrsP2-2nmrdH-1}}}\endmoddef\nwstartdeflinemarkup\nwusesondefline{\\{NW2WrsP2-3ri0xc-1}}\nwenddeflinemarkup
@pytest.fixture
def mock_canvas():
    """
    Mock Canvas API instance.

    Provides get_courses() method that returns a list of mock courses.
    Tests can customize the return value to simulate different scenarios.
    """
    canvas = MagicMock()
    canvas.get_courses.return_value = []
    return canvas
\nwused{\\{NW2WrsP2-3ri0xc-1}}\nwendcode{}\nwbegindocs{4}\nwdocspar

\subsection{Mock course objects}

Courses are central to Canvas---almost every operation starts by selecting a
course.
We provide both single-course and multi-course fixtures.

\subsubsection{Design decision: realistic vs. minimal mocks}

We face a choice: should our mock courses have every attribute a real Canvas
course has (100+ attributes), or just the ones our code uses?

We choose \emph{minimal but realistic} mocks:
\begin{itemize}
\item Include attributes our code actually accesses ({\Tt{}id\nwendquote}, {\Tt{}name\nwendquote},
{\Tt{}course{\_}code\nwendquote})
\item Provide methods our code calls ({\Tt{}get{\_}assignments()\nwendquote}, {\Tt{}get{\_}users()\nwendquote})
\item Omit rarely-used attributes to keep tests readable
\item Allow tests to add custom attributes when needed (MagicMock permits this)
\end{itemize}

This gives us the best of both worlds: tests are clear about what they're
testing, but flexible enough for edge cases.

\nwenddocs{}\nwbegincode{5}\sublabel{NW2WrsP2-4SoGVR-1}\nwmargintag{{\nwtagstyle{}\subpageref{NW2WrsP2-4SoGVR-1}}}\moddef{mock course fixtures~{\nwtagstyle{}\subpageref{NW2WrsP2-4SoGVR-1}}}\endmoddef\nwstartdeflinemarkup\nwusesondefline{\\{NW2WrsP2-3ri0xc-1}}\nwenddeflinemarkup
@pytest.fixture
def mock_course():
    """
    Mock Canvas Course object with realistic attributes.

    Attributes:
        id: Course ID (12345)
        name: Course name ("Test Course")
        course_code: Short code ("TEST101")
        term: Dict with term info

    Methods:
        get_assignments(): Returns empty list (override in tests)
        get_users(): Returns empty list (override in tests)
    """
    course = MagicMock()
    course.id = 12345
    course.name = "Test Course"
    course.course_code = "TEST101"
    course.term = \{"name": "Fall 2023"\}

    # Mock methods that return lists
    course.get_assignments.return_value = []
    course.get_users.return_value = []

    return course


@pytest.fixture
def mock_courses():
    """
    Returns a list of three mock courses with different names.

    Useful for testing filtering and selection logic.
    Uses variation theory: three courses with same structure,
    different data (names, codes, IDs).
    """
    course1 = MagicMock()
    course1.id = 1001
    course1.name = "Introduction to Programming"
    course1.course_code = "CS101"
    course1.term = \{"name": "Fall 2023"\}
    course1.get_assignments.return_value = []
    course1.get_users.return_value = []

    course2 = MagicMock()
    course2.id = 1002
    course2.name = "Advanced Algorithms"
    course2.course_code = "CS201"
    course2.term = \{"name": "Fall 2023"\}
    course2.get_assignments.return_value = []
    course2.get_users.return_value = []

    course3 = MagicMock()
    course3.id = 1003
    course3.name = "Data Structures"
    course3.course_code = "CS102"
    course3.term = \{"name": "Spring 2024"\}
    course3.get_assignments.return_value = []
    course3.get_users.return_value = []

    return [course1, course2, course3]
\nwused{\\{NW2WrsP2-3ri0xc-1}}\nwendcode{}\nwbegindocs{6}\nwdocspar

\subsection{Mock assignment objects}

Assignments represent tasks that students submit.
Our mock assignments include common attributes like due dates, point values,
and submission types.

\nwenddocs{}\nwbegincode{7}\sublabel{NW2WrsP2-3VuERV-1}\nwmargintag{{\nwtagstyle{}\subpageref{NW2WrsP2-3VuERV-1}}}\moddef{mock assignment fixtures~{\nwtagstyle{}\subpageref{NW2WrsP2-3VuERV-1}}}\endmoddef\nwstartdeflinemarkup\nwusesondefline{\\{NW2WrsP2-3ri0xc-1}}\nwenddeflinemarkup
@pytest.fixture
def mock_assignment():
    """
    Mock Canvas Assignment object.

    Attributes:
        id: Assignment ID
        name: Assignment name
        due_at: ISO 8601 due date string (or None)
        points_possible: Maximum points
        published: Whether assignment is published
        assignment_group_id: ID of assignment group
        submission_types: List of allowed submission types

    Methods:
        get_submissions(): Returns empty list (override in tests)
    """
    assignment = MagicMock()
    assignment.id = 5001
    assignment.name = "Homework 1"
    assignment.due_at = "2023-12-15T23:59:00Z"
    assignment.points_possible = 100.0
    assignment.published = True
    assignment.assignment_group_id = 2001
    assignment.submission_types = ["online_text_entry", "online_upload"]
    assignment.get_submissions.return_value = []

    return assignment


@pytest.fixture
def mock_assignment_group():
    """
    Mock Canvas AssignmentGroup object.

    Assignment groups categorize assignments (e.g., "Homeworks", "Exams").

    Attributes:
        id: Group ID
        name: Group name
        position: Sort order
        group_weight: Percentage of final grade
    """
    group = MagicMock()
    group.id = 2001
    group.name = "Homeworks"
    group.position = 1
    group.group_weight = 40.0

    return group
\nwused{\\{NW2WrsP2-3ri0xc-1}}\nwendcode{}\nwbegindocs{8}\nwdocspar

\subsection{Mock user and group objects}

Users are students, teachers, and TAs.
Groups are collections of students for group assignments.

\nwenddocs{}\nwbegincode{9}\sublabel{NW2WrsP2-htc1N-1}\nwmargintag{{\nwtagstyle{}\subpageref{NW2WrsP2-htc1N-1}}}\moddef{mock user and group fixtures~{\nwtagstyle{}\subpageref{NW2WrsP2-htc1N-1}}}\endmoddef\nwstartdeflinemarkup\nwusesondefline{\\{NW2WrsP2-3ri0xc-1}}\nwenddeflinemarkup
@pytest.fixture
def mock_user():
    """
    Mock Canvas User object.

    Attributes:
        id: User ID
        name: Full name
        sortable_name: "Last, First" format
        login_id: Username
        email: Email address (may be None due to privacy settings)
    """
    user = MagicMock()
    user.id = 3001
    user.name = "Alice Anderson"
    user.sortable_name = "Anderson, Alice"
    user.login_id = "alice"
    user.email = "alice@example.edu"

    return user


@pytest.fixture
def mock_group():
    """
    Mock Canvas Group object.

    Groups are collections of users for collaborative assignments.

    Attributes:
        id: Group ID
        name: Group name
        members_count: Number of members

    Methods:
        get_users(): Returns list of mock users
    """
    group = MagicMock()
    group.id = 4001
    group.name = "Project Team Alpha"
    group.members_count = 3
    group.get_users.return_value = []

    return group
\nwused{\\{NW2WrsP2-3ri0xc-1}}\nwendcode{}\nwbegindocs{10}\nwdocspar

\subsection{Mock submission objects}

Submissions represent student work on assignments.
They have workflow states (submitted, graded, etc.) and may include scores,
comments, and file attachments.

\nwenddocs{}\nwbegincode{11}\sublabel{NW2WrsP2-qDrc3-1}\nwmargintag{{\nwtagstyle{}\subpageref{NW2WrsP2-qDrc3-1}}}\moddef{mock submission fixtures~{\nwtagstyle{}\subpageref{NW2WrsP2-qDrc3-1}}}\endmoddef\nwstartdeflinemarkup\nwusesondefline{\\{NW2WrsP2-3ri0xc-1}}\nwenddeflinemarkup
@pytest.fixture
def mock_submission():
    """
    Mock Canvas Submission object.

    Submissions represent student work. They have states (submitted, graded,
    unsubmitted) and may include scores, comments, and timestamps.

    Attributes:
        id: Submission ID
        user_id: ID of student who submitted
        assignment_id: ID of assignment
        workflow_state: State (submitted, graded, unsubmitted, pending_review)
        grade: Current grade (may be None)
        score: Numeric score (may be None)
        submitted_at: ISO 8601 submission time (or None)
        graded_at: ISO 8601 grading time (or None)
        late: Whether submission was late
        missing: Whether submission is missing
        excused: Whether student is excused
    """
    submission = MagicMock()
    submission.id = 6001
    submission.user_id = 3001
    submission.assignment_id = 5001
    submission.workflow_state = "submitted"
    submission.grade = "A"
    submission.score = 95.0
    submission.submitted_at = "2023-12-14T18:30:00Z"
    submission.graded_at = "2023-12-16T10:00:00Z"
    submission.late = False
    submission.missing = False
    submission.excused = False

    return submission
\nwused{\\{NW2WrsP2-3ri0xc-1}}\nwendcode{}\nwbegindocs{12}\nwdocspar

\subsection{Mock configuration}

Many functions need the configuration dictionary that's normally loaded from
{\Tt{}{\char126}/.config/canvaslms/config.json\nwendquote}.
We provide a fixture with sensible defaults.

\nwenddocs{}\nwbegincode{13}\sublabel{NW2WrsP2-2fL3Kj-1}\nwmargintag{{\nwtagstyle{}\subpageref{NW2WrsP2-2fL3Kj-1}}}\moddef{mock configuration fixture~{\nwtagstyle{}\subpageref{NW2WrsP2-2fL3Kj-1}}}\endmoddef\nwstartdeflinemarkup\nwusesondefline{\\{NW2WrsP2-3ri0xc-1}}\nwenddeflinemarkup
@pytest.fixture
def mock_config():
    """
    Mock configuration dictionary.

    Provides minimal configuration needed for most commands.
    Tests can override values as needed.

    Keys:
        canvas_server: Canvas instance URL
        course_id: Default course ID (optional)
        assignment: Default assignment filter (optional)
    """
    return \{
        "canvas_server": "https://canvas.example.edu",
        "course_id": None,
        "assignment": None,
    \}
\nwused{\\{NW2WrsP2-3ri0xc-1}}\nwendcode{}

\nwixlogsorted{c}{{conftest.py}{NW2WrsP2-3ri0xc-1}{\nwixd{NW2WrsP2-3ri0xc-1}}}%
\nwixlogsorted{c}{{mock assignment fixtures}{NW2WrsP2-3VuERV-1}{\nwixu{NW2WrsP2-3ri0xc-1}\nwixd{NW2WrsP2-3VuERV-1}}}%
\nwixlogsorted{c}{{mock canvas api fixture}{NW2WrsP2-2nmrdH-1}{\nwixu{NW2WrsP2-3ri0xc-1}\nwixd{NW2WrsP2-2nmrdH-1}}}%
\nwixlogsorted{c}{{mock configuration fixture}{NW2WrsP2-2fL3Kj-1}{\nwixu{NW2WrsP2-3ri0xc-1}\nwixd{NW2WrsP2-2fL3Kj-1}}}%
\nwixlogsorted{c}{{mock course fixtures}{NW2WrsP2-4SoGVR-1}{\nwixu{NW2WrsP2-3ri0xc-1}\nwixd{NW2WrsP2-4SoGVR-1}}}%
\nwixlogsorted{c}{{mock submission fixtures}{NW2WrsP2-qDrc3-1}{\nwixu{NW2WrsP2-3ri0xc-1}\nwixd{NW2WrsP2-qDrc3-1}}}%
\nwixlogsorted{c}{{mock user and group fixtures}{NW2WrsP2-htc1N-1}{\nwixu{NW2WrsP2-3ri0xc-1}\nwixd{NW2WrsP2-htc1N-1}}}%
\nwbegindocs{14}\nwdocspar

\section{Using these fixtures in tests}

With these fixtures in place, writing tests becomes straightforward.
Here's a preview of how they'll be used:

\begin{verbatim}
def test_filter_courses_by_name(mock_canvas, mock_courses):
    """Test that regex filtering works on course names"""
    mock_canvas.get_courses.return_value = mock_courses

    # Test the actual filtering function
    result = filter_courses(mock_canvas, "Programming")

    assert len(result) == 1
    assert result[0].name == "Introduction to Programming"
\end{verbatim}

Notice:
\begin{itemize}
\item We declare which fixtures we need as function parameters
\item Pytest automatically provides them
\item We customize the mock behavior ({\Tt{}get{\_}courses.return{\_}value\nwendquote})
\item We test our actual function against realistic mock data
\item The test is clear, focused, and fast
\end{itemize}

\section{Benefits of this approach}

This fixture-based testing strategy provides several advantages:

\begin{description}
\item[Reusability:] Every test can use these fixtures without duplication
\item[Consistency:] All tests use the same realistic mock objects
\item[Maintainability:] If Canvas API changes, we update fixtures in one place
\item[Readability:] Tests focus on behavior, not mock setup
\item[Pedagogical value:] Fixtures serve as documentation of Canvas API
structure
\end{description}

As you read the test examples in subsequent chapters, you'll see these fixtures
in action.
They make tests clear and concise while providing enough realism to catch bugs.

\section{Running the tests}

The {\Tt{}canvaslms\nwendquote} project uses Poetry for dependency management.
Test dependencies ({\Tt{}pytest\nwendquote}, {\Tt{}pytest-cov\nwendquote}, {\Tt{}freezegun\nwendquote}, {\Tt{}pytest-mock\nwendquote}, {\Tt{}responses\nwendquote}) are installed in Poetry's virtual environment.

To run tests, you must use {\Tt{}poetry\ run\ pytest\nwendquote}:

\begin{verbatim}
# Tangle test files from .nw sources
cd tests
make all

# Run all tests with coverage
poetry run pytest -v --cov=canvaslms --cov-report=term-missing

# Or use the make target (which calls poetry run pytest)
make test
\end{verbatim}

\textbf{Important:} Simply running {\Tt{}pytest\nwendquote} without {\Tt{}poetry\ run\nwendquote} will fail because pytest and its plugins won't be found in the system path.
They exist only in Poetry's virtual environment.

The build system integrates testing:
\begin{itemize}
\item {\Tt{}make\ all\nwendquote} in {\Tt{}tests/\nwendquote} tangles test files from .nw sources
\item {\Tt{}make\ test\nwendquote} in {\Tt{}tests/\nwendquote} runs {\Tt{}poetry\ run\ pytest\nwendquote} with coverage
\item {\Tt{}make\ all\nwendquote} in the root directory compiles, builds documentation, and runs all tests
\end{itemize}

\section{Next steps}

In the following chapters, we'll see these fixtures used to test:
\begin{itemize}
\item Time formatting utilities (with {\Tt{}freezegun\nwendquote} for deterministic time
tests)
\item Course filtering with regex patterns
\item Assignment and user filtering
\item Submission processing and state management
\item Canvas API extensions (equality, hashing)
\end{itemize}

Each test chapter will show the \emph{expected behavior} through test examples
before diving into the implementation details.
This "tests as examples" approach follows variation theory: we show what the
code \emph{should do} before explaining how it \emph{does} it.
\nwenddocs{}
