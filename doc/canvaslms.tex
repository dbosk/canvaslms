\documentclass[a4paper,oneside]{book}
\newenvironment{abstract}{}{}
\usepackage{noweb}
% Needed to relax penalty for breaking code chunks across pages, otherwise 
% there might be a lot of space following a code chunk.
\def\nwendcode{\endtrivlist \endgroup}
\let\nwdocspar=\smallbreak

\usepackage[hyphens]{url}
\usepackage{hyperref}
\usepackage{authblk}

\usepackage[utf8]{inputenc}
\usepackage[T1]{fontenc}
\usepackage[british]{babel}
\usepackage{booktabs}

\usepackage[natbib,style=alphabetic,maxbibnames=99]{biblatex}
\addbibresource{canvas.bib}

\usepackage[all]{foreign}
\renewcommand{\foreignfullfont}{}
\renewcommand{\foreignabbrfont}{}

\usepackage{newclude}
\usepackage{import}

\usepackage[strict]{csquotes}
\usepackage[single]{acro}

\usepackage{subcaption}

\usepackage[noend]{algpseudocode}
\usepackage{xparse}

\let\email\texttt

\usepackage[outputdir=ltxobj]{minted}
\setminted{autogobble,linenos}

\usepackage{pythontex}
\setpythontexoutputdir{.}
\setpythontexworkingdir{..}

\usepackage{fancyvrb}

\usepackage{amsmath}
\usepackage{amssymb}
\usepackage{mathtools}
\usepackage{amsthm}
\usepackage{thmtools}
\usepackage[unq]{unique}
\DeclareMathOperator{\powerset}{\mathcal{P}}

\usepackage[binary-units]{siunitx}

\usepackage[capitalize]{cleveref}


\title{%
  Canvas LMS: a command line interface
}
\author{%
  Daniel Bosk
}
\affil{%
  KTH EECS
}

\begin{document}
\frontmatter
\maketitle

\vspace*{\fill}
\VerbatimInput{../LICENSE}
\clearpage

\begin{abstract}
  % What's the problem?
% Why is it a problem? Research gap left by other approaches?
% Why is it important? Why care?
% What's the approach? How to solve the problem?
% What's the findings? How was it evaluated, what are the results, limitations, 
% what remains to be done?

This program provides a command-line interface for Canvas.
The command is \texttt{canvas} and it has several subcommands in the same style 
as Git.
\texttt{canvas} provides output in a format useful for POSIX tools, this makes 
automating tasks much easier.

Let's consider how to grade students logging into the student-shell SSH server.
We store the list of students' Canvas and KTH IDs in a file.
\begin{lstlisting}
canvas users -c DD1301 -s | cut -f 1,2 > students.tsv
\end{lstlisting}
Then we check who has logged into student-shell.
\begin{lstlisting}[firstnumber=2]
ssh student-shell.sys.kth.se last | cut -f 1 -d " " | sort | uniq \
  > logged-in.tsv
\end{lstlisting}
Finally, we check who of our students logged in.
\begin{lstlisting}[firstnumber=4]
for s in $(cut -f 2 students.tsv); do
  grep $s logged-in.tsv && \
\end{lstlisting}
Finally, we can set their grade to P and add the comment \enquote{Well done!} 
in Canvas.
We set the grades for the two assignments whose titles match the regular 
expression \texttt{(Preparing the terminal|The terminal)}.
\begin{lstlisting}[firstnumber=6]
    canvas grade -c DD1301 -a "(Preparing the terminal|The terminal)" \
      -u $(grep $s students.tsv | cut -f 1) \
      -g P -m "Well done!"
done
\end{lstlisting}


\end{abstract}
\clearpage


\tableofcontents
\clearpage

\mainmatter


\chapter{Introduction}

This is the documentation of the \texttt{canvaslms} Python package and the 
\texttt{canvaslms} command.
It provides a command-line interface for the Canvas LMS\@.

The command comes as a PyPI package.
It can be installed by running:
\begin{minted}{text}
  python3 -m pip install canvaslms
\end{minted}
Then you can use \texttt{canvaslms -h} for further usage instructions.

Some subcommands use the \texttt{pandoc} command\footnote{%
  URL: \url{https://pandoc.org/installing.html}.
}.
You will have to install that command manually.


\section{For contributors}

The package is divided into modules.
The \texttt{cli} moduile provides the base for the command and it uses the 
other modules.
There is a module for each subcommand.
Each such module must provide a
function~\texttt{add\textunderscore{}command\textunderscore{}options} which 
takes an \texttt{argparse} (sub)parser as an argument.
The subcommand must add a function which takes three arguments:
\begin{enumerate}
  \item \texttt{config}, a dictionary with the configuration data.
  \item \texttt{canvas}, a \texttt{canvasapi} object.
  \item \texttt{args}, the arguments parsed by \texttt{argparse}.
\end{enumerate}


%%% THE SOURCES %%%

\input{../src/canvaslms/cli/cli.tex}
\input{../src/canvaslms/cli/courses.tex}
\input{../src/canvaslms/cli/users.tex}
\input{../src/canvaslms/cli/assignments.tex}
%\input{../src/canvaslms/cli/submissions.tex}
%\input{../src/canvaslms/cli/grades.tex}


\printbibliography
\end{document}
