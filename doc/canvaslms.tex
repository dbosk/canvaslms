\documentclass[a4paper,oneside]{memoir}
\usepackage{noweb}
\noweboptions{breakcode,longxref,longchunks}

\usepackage[hyphens]{url}
\usepackage[colorlinks]{hyperref}
\usepackage{authblk}

\usepackage[utf8]{inputenc}
\usepackage[T1]{fontenc}
\usepackage[british]{babel}
\usepackage{booktabs}

\usepackage[natbib,style=alphabetic,maxbibnames=99]{biblatex}
\addbibresource{canvas.bib}

\usepackage[all]{foreign}
\renewcommand{\foreignfullfont}{}
\renewcommand{\foreignabbrfont}{}

\usepackage{newclude}
\usepackage{import}

\usepackage[strict]{csquotes}
\usepackage[single]{acro}

\usepackage{subcaption}

\usepackage[noend]{algpseudocode}
\usepackage{xparse}

\let\email\texttt

\usepackage[outputdir=ltxobj]{minted}
\setminted{autogobble,linenos}

\usepackage{pythontex}
\setpythontexoutputdir{.}
\setpythontexworkingdir{..}

\usepackage{fancyvrb}

\usepackage{amsmath}
\usepackage{amssymb}
\usepackage{mathtools}
\usepackage{amsthm}
\usepackage{thmtools}
\usepackage[unq]{unique}
\DeclareMathOperator{\powerset}{\mathcal{P}}

\usepackage[binary-units]{siunitx}

\usepackage[capitalize]{cleveref}

\usepackage{didactic}

\title{%
  Canvas LMS: a command line interface
}
\author{%
  Daniel Bosk
}
\affil{%
  KTH EECS
}

\begin{document}
\frontmatter
\maketitle

\vspace*{\fill}
\VerbatimInput{../LICENSE}

\clearpage
% What's the problem?
% Why is it a problem? Research gap left by other approaches?
% Why is it important? Why care?
% What's the approach? How to solve the problem?
% What's the findings? How was it evaluated, what are the results, limitations, 
% what remains to be done?

This program provides a command-line interface for Canvas.
The command is \texttt{canvas} and it has several subcommands in the same style 
as Git.
\texttt{canvas} provides output in a format useful for POSIX tools, this makes 
automating tasks much easier.

Let's consider how to grade students logging into the student-shell SSH server.
We store the list of students' Canvas and KTH IDs in a file.
\begin{lstlisting}
canvas users -c DD1301 -s | cut -f 1,2 > students.tsv
\end{lstlisting}
Then we check who has logged into student-shell.
\begin{lstlisting}[firstnumber=2]
ssh student-shell.sys.kth.se last | cut -f 1 -d " " | sort | uniq \
  > logged-in.tsv
\end{lstlisting}
Finally, we check who of our students logged in.
\begin{lstlisting}[firstnumber=4]
for s in $(cut -f 2 students.tsv); do
  grep $s logged-in.tsv && \
\end{lstlisting}
Finally, we can set their grade to P and add the comment \enquote{Well done!} 
in Canvas.
We set the grades for the two assignments whose titles match the regular 
expression \texttt{(Preparing the terminal|The terminal)}.
\begin{lstlisting}[firstnumber=6]
    canvas grade -c DD1301 -a "(Preparing the terminal|The terminal)" \
      -u $(grep $s students.tsv | cut -f 1) \
      -g P -m "Well done!"
done
\end{lstlisting}


\clearpage


\tableofcontents
\clearpage

\mainmatter


\chapter{Introduction}

This is the documentation of the \texttt{canvaslms} Python package and the 
\texttt{canvaslms} command.
It provides a command-line interface for the Canvas LMS\@.

The command comes as a PyPI package.
It can be installed by running:
\begin{minted}{text}
  python3 -m pip install canvaslms
\end{minted}
Then you can use \texttt{canvaslms -h} for further usage instructions.

Some subcommands use the \texttt{pandoc} command\footnote{%
  URL: \url{https://pandoc.org/installing.html}.
}.
You will have to install that command manually.


\section{For contributors}

The package is divided into modules.
The \mintinline{python}|cli| module provides the base for the command and it uses the 
other modules.
There is a module for each subcommand.
Each such module must provide a function~\mintinline{python}|add_command| which takes 
an \mintinline{python}|argparse| (sub)parser as an argument.
\begin{minted}{python}
  def add_command(subp):
    command_parser = subp.add_parser("command", ...)
    command_parser.set_defaults(func=command_function)
\end{minted}
The subcommand must add the function~\mintinline{python}|command_function| which 
takes three arguments:
\begin{enumerate}
  \item \mintinline{python}|config|, a dictionary with the configuration data.
  \item \mintinline{python}|canvas|, a \mintinline{python}|canvasapi| object.
  \item \mintinline{python}|args|, the arguments parsed by 
    \mintinline{python}|argparse|.
\end{enumerate}
Something along these lines:
\begin{minted}{python}
def command_function(config, canvas, args):
  # process the args and do its thing
\end{minted}


%%% THE SOURCES %%%

\part{Some hacks}

\input{../src/canvaslms/hacks/canvasapi.tex}

\part{The command-line interface}

\input{../src/canvaslms/cli/cli.tex}
\input{../src/canvaslms/cli/login.tex}
\input{../src/canvaslms/cli/courses.tex}
\input{../src/canvaslms/cli/users.tex}
\input{../src/canvaslms/cli/assignments.tex}
\input{../src/canvaslms/cli/submissions.tex}
\input{../src/canvaslms/cli/grade.tex}
\input{../src/canvaslms/cli/results.tex}
\input{../src/canvaslms/grades/grades.tex}
\chapter{The \texttt{calendar} commands}% ===> this file was generated automatically by noweave --- better not edit it

This chapter provides calendar-related subcommands for Canvas LMS.
The calendar commands allow users to list, view, and create calendar events.

Canvas provides several calendar-related API endpoints:
\begin{itemize}
\item \texttt{get\_calendar\_events} to list calendar events
\item \texttt{get\_calendar\_event} to get a specific calendar event
\item \texttt{create\_calendar\_event} to create new calendar events
\item \texttt{get\_upcoming\_events} to get upcoming events
\end{itemize}

\section{Module outline}

We outline the module:
\nwfilename{calendar.nw}\nwbegincode{1}\sublabel{NW2B3lYx-sDzND-1}\nwmargintag{{\nwtagstyle{}\subpageref{NW2B3lYx-sDzND-1}}}\moddef{calendar.py~{\nwtagstyle{}\subpageref{NW2B3lYx-sDzND-1}}}\endmoddef\nwstartdeflinemarkup\nwenddeflinemarkup
import argparse
import canvasapi.exceptions
import canvaslms.cli
import canvaslms.cli.courses as courses
import csv
import datetime
import sys
import arrow

\LA{}functions~{\nwtagstyle{}\subpageref{NW2B3lYx-nRuDO-1}}\RA{}

def add_command(subp):
  """Adds the calendar subcommands to argparse subparser subp"""
  add_calendar_list_command(subp)
  add_calendar_show_command(subp)
  add_calendar_create_command(subp)
\nwnotused{calendar.py}\nwendcode{}\nwbegindocs{2}\nwdocspar


\section{The \texttt{calendar list} command}

The \texttt{calendar list} command lists calendar events, optionally filtered by course.
\nwenddocs{}\nwbegincode{3}\sublabel{NW2B3lYx-nRuDO-1}\nwmargintag{{\nwtagstyle{}\subpageref{NW2B3lYx-nRuDO-1}}}\moddef{functions~{\nwtagstyle{}\subpageref{NW2B3lYx-nRuDO-1}}}\endmoddef\nwstartdeflinemarkup\nwusesondefline{\\{NW2B3lYx-sDzND-1}}\nwprevnextdefs{\relax}{NW2B3lYx-nRuDO-2}\nwenddeflinemarkup
def add_calendar_list_command(subp):
  """Adds the calendar list subcommand and its options to argparse subparser subp"""
  calendar_list_parser = subp.add_parser("calendar-list",
    help="Lists calendar events",
    description="Lists calendar events. Output, CSV-format: "
      "<event-id> <title> <start-time> <end-time> <context-type> <context-name>")
  calendar_list_parser.set_defaults(func=calendar_list_command)
  courses.add_course_option(calendar_list_parser)
  \LA{}add calendar list arguments~{\nwtagstyle{}\subpageref{NW2B3lYx-2ZPLrI-1}}\RA{}

def calendar_list_command(config, canvas, args):
  """Lists calendar events in CSV format to stdout"""
  output = csv.writer(sys.stdout, delimiter=args.delimiter)
  
  \LA{}process course option for calendar list~{\nwtagstyle{}\subpageref{NW2B3lYx-FIYeH-1}}\RA{}
  \LA{}get and output calendar events~{\nwtagstyle{}\subpageref{NW2B3lYx-32Uaup-1}}\RA{}
\nwalsodefined{\\{NW2B3lYx-nRuDO-2}\\{NW2B3lYx-nRuDO-3}}\nwused{\\{NW2B3lYx-sDzND-1}}\nwendcode{}\nwbegindocs{4}\nwdocspar

We add options to filter events by date range.
\nwenddocs{}\nwbegincode{5}\sublabel{NW2B3lYx-2ZPLrI-1}\nwmargintag{{\nwtagstyle{}\subpageref{NW2B3lYx-2ZPLrI-1}}}\moddef{add calendar list arguments~{\nwtagstyle{}\subpageref{NW2B3lYx-2ZPLrI-1}}}\endmoddef\nwstartdeflinemarkup\nwusesondefline{\\{NW2B3lYx-nRuDO-1}}\nwenddeflinemarkup
calendar_list_parser.add_argument("--start-date",
  help="Start date for event filter (YYYY-MM-DD format)")
calendar_list_parser.add_argument("--end-date", 
  help="End date for event filter (YYYY-MM-DD format)")
calendar_list_parser.add_argument("--type",
  choices=["event", "assignment"],
  help="Filter by event type")
\nwused{\\{NW2B3lYx-nRuDO-1}}\nwendcode{}\nwbegindocs{6}\nwdocspar


\section{Processing course options for calendar events}

If a course is specified, we get events for that course context.
Otherwise, we get all events for the user.
\nwenddocs{}\nwbegincode{7}\sublabel{NW2B3lYx-FIYeH-1}\nwmargintag{{\nwtagstyle{}\subpageref{NW2B3lYx-FIYeH-1}}}\moddef{process course option for calendar list~{\nwtagstyle{}\subpageref{NW2B3lYx-FIYeH-1}}}\endmoddef\nwstartdeflinemarkup\nwusesondefline{\\{NW2B3lYx-nRuDO-1}}\nwenddeflinemarkup
context_codes = []
if hasattr(args, 'course') and args.course:
  course_list = courses.process_course_option(canvas, args)
  context_codes = [f"course_\{course.id\}" for course in course_list]
\nwused{\\{NW2B3lYx-nRuDO-1}}\nwendcode{}\nwbegindocs{8}\nwdocspar


\section{Getting and outputting calendar events}

We fetch the calendar events and output them in CSV format.
\nwenddocs{}\nwbegincode{9}\sublabel{NW2B3lYx-32Uaup-1}\nwmargintag{{\nwtagstyle{}\subpageref{NW2B3lYx-32Uaup-1}}}\moddef{get and output calendar events~{\nwtagstyle{}\subpageref{NW2B3lYx-32Uaup-1}}}\endmoddef\nwstartdeflinemarkup\nwusesondefline{\\{NW2B3lYx-nRuDO-1}}\nwenddeflinemarkup
try:
  # Prepare parameters for API call
  params = \{\}
  if context_codes:
    params['context_codes'] = context_codes
  if args.start_date:
    params['start_date'] = args.start_date
  if args.end_date:
    params['end_date'] = args.end_date
  if args.type:
    params['type'] = args.type
    
  events = canvas.get_calendar_events(**params)
  
  for event in events:
    \LA{}output calendar event data~{\nwtagstyle{}\subpageref{NW2B3lYx-1CSmd7-1}}\RA{}
    
except canvasapi.exceptions.CanvasException as e:
  canvaslms.cli.err(1, f"Failed to get calendar events: \{e\}")
\nwused{\\{NW2B3lYx-nRuDO-1}}\nwendcode{}\nwbegindocs{10}\nwdocspar

We output the calendar event data in a structured format.
\nwenddocs{}\nwbegincode{11}\sublabel{NW2B3lYx-1CSmd7-1}\nwmargintag{{\nwtagstyle{}\subpageref{NW2B3lYx-1CSmd7-1}}}\moddef{output calendar event data~{\nwtagstyle{}\subpageref{NW2B3lYx-1CSmd7-1}}}\endmoddef\nwstartdeflinemarkup\nwusesondefline{\\{NW2B3lYx-32Uaup-1}}\nwenddeflinemarkup
# Get context information
context_type = getattr(event, 'context_type', '')
context_name = getattr(event, 'context_name', '')

# Format dates
start_at = getattr(event, 'start_at', '')
end_at = getattr(event, 'end_at', '')

row = [
  getattr(event, 'id', ''),
  getattr(event, 'title', ''),
  start_at,
  end_at,
  context_type,
  context_name
]
output.writerow(row)
\nwused{\\{NW2B3lYx-32Uaup-1}}\nwendcode{}\nwbegindocs{12}\nwdocspar


\section{The \texttt{calendar show} command}

The \texttt{calendar show} command shows details of a specific calendar event.
\nwenddocs{}\nwbegincode{13}\sublabel{NW2B3lYx-nRuDO-2}\nwmargintag{{\nwtagstyle{}\subpageref{NW2B3lYx-nRuDO-2}}}\moddef{functions~{\nwtagstyle{}\subpageref{NW2B3lYx-nRuDO-1}}}\plusendmoddef\nwstartdeflinemarkup\nwusesondefline{\\{NW2B3lYx-sDzND-1}}\nwprevnextdefs{NW2B3lYx-nRuDO-1}{NW2B3lYx-nRuDO-3}\nwenddeflinemarkup
def add_calendar_show_command(subp):
  """Adds the calendar show subcommand and its options to argparse subparser subp"""
  calendar_show_parser = subp.add_parser("calendar-show",
    help="Shows details of a specific calendar event",
    description="Shows details of a specific calendar event")
  calendar_show_parser.set_defaults(func=calendar_show_command)
  \LA{}add calendar show arguments~{\nwtagstyle{}\subpageref{NW2B3lYx-3sJ4bu-1}}\RA{}

def calendar_show_command(config, canvas, args):
  """Shows details of a specific calendar event"""
  \LA{}get and display calendar event details~{\nwtagstyle{}\subpageref{NW2B3lYx-4MmSaH-1}}\RA{}
\nwused{\\{NW2B3lYx-sDzND-1}}\nwendcode{}\nwbegindocs{14}\nwdocspar

The show command requires an event ID.
\nwenddocs{}\nwbegincode{15}\sublabel{NW2B3lYx-3sJ4bu-1}\nwmargintag{{\nwtagstyle{}\subpageref{NW2B3lYx-3sJ4bu-1}}}\moddef{add calendar show arguments~{\nwtagstyle{}\subpageref{NW2B3lYx-3sJ4bu-1}}}\endmoddef\nwstartdeflinemarkup\nwusesondefline{\\{NW2B3lYx-nRuDO-2}}\nwenddeflinemarkup
calendar_show_parser.add_argument("event_id",
  help="The ID of the calendar event to show")
\nwused{\\{NW2B3lYx-nRuDO-2}}\nwendcode{}\nwbegindocs{16}\nwdocspar

We fetch and display the event details.
\nwenddocs{}\nwbegincode{17}\sublabel{NW2B3lYx-4MmSaH-1}\nwmargintag{{\nwtagstyle{}\subpageref{NW2B3lYx-4MmSaH-1}}}\moddef{get and display calendar event details~{\nwtagstyle{}\subpageref{NW2B3lYx-4MmSaH-1}}}\endmoddef\nwstartdeflinemarkup\nwusesondefline{\\{NW2B3lYx-nRuDO-2}}\nwenddeflinemarkup
try:
  event = canvas.get_calendar_event(args.event_id)
  
  print(f"Event ID: \{event.id\}")
  print(f"Title: \{getattr(event, 'title', 'N/A')\}")
  print(f"Description: \{getattr(event, 'description', 'N/A')\}")
  print(f"Start: \{getattr(event, 'start_at', 'N/A')\}")
  print(f"End: \{getattr(event, 'end_at', 'N/A')\}")
  print(f"Location: \{getattr(event, 'location_name', 'N/A')\}")
  print(f"Context: \{getattr(event, 'context_type', 'N/A')\} - \{getattr(event, 'context_name', 'N/A')\}")
  print(f"URL: \{getattr(event, 'html_url', 'N/A')\}")
  
except canvasapi.exceptions.CanvasException as e:
  canvaslms.cli.err(1, f"Failed to get calendar event: \{e\}")
\nwused{\\{NW2B3lYx-nRuDO-2}}\nwendcode{}\nwbegindocs{18}\nwdocspar


\section{The \texttt{calendar create} command}

The \texttt{calendar create} command creates a new calendar event.
\nwenddocs{}\nwbegincode{19}\sublabel{NW2B3lYx-nRuDO-3}\nwmargintag{{\nwtagstyle{}\subpageref{NW2B3lYx-nRuDO-3}}}\moddef{functions~{\nwtagstyle{}\subpageref{NW2B3lYx-nRuDO-1}}}\plusendmoddef\nwstartdeflinemarkup\nwusesondefline{\\{NW2B3lYx-sDzND-1}}\nwprevnextdefs{NW2B3lYx-nRuDO-2}{\relax}\nwenddeflinemarkup
def add_calendar_create_command(subp):
  """Adds the calendar create subcommand and its options to argparse subparser subp"""  
  calendar_create_parser = subp.add_parser("calendar-create",
    help="Creates a new calendar event", 
    description="Creates a new calendar event")
  calendar_create_parser.set_defaults(func=calendar_create_command)
  courses.add_course_option(calendar_create_parser)
  \LA{}add calendar create arguments~{\nwtagstyle{}\subpageref{NW2B3lYx-1EjBJB-1}}\RA{}

def calendar_create_command(config, canvas, args):
  """Creates a new calendar event"""
  \LA{}create calendar event~{\nwtagstyle{}\subpageref{NW2B3lYx-4DP8GG-1}}\RA{}
\nwused{\\{NW2B3lYx-sDzND-1}}\nwendcode{}\nwbegindocs{20}\nwdocspar

The create command requires title and start time.
\nwenddocs{}\nwbegincode{21}\sublabel{NW2B3lYx-1EjBJB-1}\nwmargintag{{\nwtagstyle{}\subpageref{NW2B3lYx-1EjBJB-1}}}\moddef{add calendar create arguments~{\nwtagstyle{}\subpageref{NW2B3lYx-1EjBJB-1}}}\endmoddef\nwstartdeflinemarkup\nwusesondefline{\\{NW2B3lYx-nRuDO-3}}\nwenddeflinemarkup
calendar_create_parser.add_argument("title",
  help="Title of the calendar event")
calendar_create_parser.add_argument("--start-time", required=True,
  help="Start time (ISO format: YYYY-MM-DDTHH:MM:SS)")
calendar_create_parser.add_argument("--end-time",
  help="End time (ISO format: YYYY-MM-DDTHH:MM:SS)")
calendar_create_parser.add_argument("--description", 
  help="Description of the event")
calendar_create_parser.add_argument("--location",
  help="Location of the event")
\nwused{\\{NW2B3lYx-nRuDO-3}}\nwendcode{}\nwbegindocs{22}\nwdocspar

We create the calendar event with the specified parameters.
\nwenddocs{}\nwbegincode{23}\sublabel{NW2B3lYx-4DP8GG-1}\nwmargintag{{\nwtagstyle{}\subpageref{NW2B3lYx-4DP8GG-1}}}\moddef{create calendar event~{\nwtagstyle{}\subpageref{NW2B3lYx-4DP8GG-1}}}\endmoddef\nwstartdeflinemarkup\nwusesondefline{\\{NW2B3lYx-nRuDO-3}}\nwenddeflinemarkup
try:
  # Prepare event parameters
  params = \{
    'calendar_event': \{
      'title': args.title,
      'start_at': args.start_time,
    \}
  \}
  
  if args.end_time:
    params['calendar_event']['end_at'] = args.end_time
  if args.description:
    params['calendar_event']['description'] = args.description  
  if args.location:
    params['calendar_event']['location_name'] = args.location
    
  # Set context if course is specified
  if hasattr(args, 'course') and args.course:
    course_list = courses.process_course_option(canvas, args)
    if course_list:
      course = list(course_list)[0]  # Use first course if multiple
      params['calendar_event']['context_code'] = f"course_\{course.id\}"
  
  event = canvas.create_calendar_event(**params)
  
  print(f"Created calendar event with ID: \{event.id\}")
  print(f"Title: \{event.title\}")
  print(f"Start: \{getattr(event, 'start_at', 'N/A')\}")
  if hasattr(event, 'end_at') and event.end_at:
    print(f"End: \{event.end_at\}")
  
except canvasapi.exceptions.CanvasException as e:
  canvaslms.cli.err(1, f"Failed to create calendar event: \{e\}")
\nwused{\\{NW2B3lYx-nRuDO-3}}\nwendcode{}

\nwixlogsorted{c}{{add calendar create arguments}{NW2B3lYx-1EjBJB-1}{\nwixu{NW2B3lYx-nRuDO-3}\nwixd{NW2B3lYx-1EjBJB-1}}}%
\nwixlogsorted{c}{{add calendar list arguments}{NW2B3lYx-2ZPLrI-1}{\nwixu{NW2B3lYx-nRuDO-1}\nwixd{NW2B3lYx-2ZPLrI-1}}}%
\nwixlogsorted{c}{{add calendar show arguments}{NW2B3lYx-3sJ4bu-1}{\nwixu{NW2B3lYx-nRuDO-2}\nwixd{NW2B3lYx-3sJ4bu-1}}}%
\nwixlogsorted{c}{{calendar.py}{NW2B3lYx-sDzND-1}{\nwixd{NW2B3lYx-sDzND-1}}}%
\nwixlogsorted{c}{{create calendar event}{NW2B3lYx-4DP8GG-1}{\nwixu{NW2B3lYx-nRuDO-3}\nwixd{NW2B3lYx-4DP8GG-1}}}%
\nwixlogsorted{c}{{functions}{NW2B3lYx-nRuDO-1}{\nwixu{NW2B3lYx-sDzND-1}\nwixd{NW2B3lYx-nRuDO-1}\nwixd{NW2B3lYx-nRuDO-2}\nwixd{NW2B3lYx-nRuDO-3}}}%
\nwixlogsorted{c}{{get and display calendar event details}{NW2B3lYx-4MmSaH-1}{\nwixu{NW2B3lYx-nRuDO-2}\nwixd{NW2B3lYx-4MmSaH-1}}}%
\nwixlogsorted{c}{{get and output calendar events}{NW2B3lYx-32Uaup-1}{\nwixu{NW2B3lYx-nRuDO-1}\nwixd{NW2B3lYx-32Uaup-1}}}%
\nwixlogsorted{c}{{output calendar event data}{NW2B3lYx-1CSmd7-1}{\nwixu{NW2B3lYx-32Uaup-1}\nwixd{NW2B3lYx-1CSmd7-1}}}%
\nwixlogsorted{c}{{process course option for calendar list}{NW2B3lYx-FIYeH-1}{\nwixu{NW2B3lYx-nRuDO-1}\nwixd{NW2B3lYx-FIYeH-1}}}%
\nwbegindocs{24}\nwdocspar
\nwenddocs{}



\printbibliography
\end{document}
