\chapter{The \texttt{announce} subcommand}% ===> this file was generated automatically by noweave --- better not edit it

This chapter provides the \texttt{announce} subcommand which allows posting
announcements to multiple courses based on a regex pattern.

We outline the module:
\nwfilename{announce.nw}\nwbegincode{1}\sublabel{NW3wjrKS-2c3aOu-1}\nwmargintag{{\nwtagstyle{}\subpageref{NW3wjrKS-2c3aOu-1}}}\moddef{announce.py~{\nwtagstyle{}\subpageref{NW3wjrKS-2c3aOu-1}}}\endmoddef\nwstartdeflinemarkup\nwenddeflinemarkup
import os
import re
import subprocess
import sys
import tempfile
import canvaslms.cli.courses

\LA{}functions~{\nwtagstyle{}\subpageref{NW3wjrKS-nRuDO-1}}\RA{}

def add_command(subp):
  """Adds the announce command to argparse parser"""
  announce_parser = subp.add_parser("announce",
    help="Post announcements to courses",
    description="Post announcements to one or more courses matching a regex pattern. "
                "The announcement can be provided via command line or interactively using an editor.")
  announce_parser.set_defaults(func=announce_command)
  \LA{}add arguments~{\nwtagstyle{}\subpageref{NW3wjrKS-Z37aL-1}}\RA{}
\nwnotused{announce.py}\nwendcode{}\nwbegindocs{2}\nwdocspar


\section{Command arguments}

The announce command requires a title and supports various ways to specify the message.
\nwenddocs{}\nwbegincode{3}\sublabel{NW3wjrKS-Z37aL-1}\nwmargintag{{\nwtagstyle{}\subpageref{NW3wjrKS-Z37aL-1}}}\moddef{add arguments~{\nwtagstyle{}\subpageref{NW3wjrKS-Z37aL-1}}}\endmoddef\nwstartdeflinemarkup\nwusesondefline{\\{NW3wjrKS-2c3aOu-1}}\nwenddeflinemarkup
announce_parser.add_argument("title",
  help="Title of the announcement")
announce_parser.add_argument("-c", "--course", required=True,
  help="Regex matching courses on title, course code or Canvas ID")
announce_parser.add_argument("-m", "--message",
  help="Message content of the announcement (use -i for interactive mode)")
announce_parser.add_argument("-i", "--interactive",
  action="store_true", default=False,
  help="Interactive mode: open editor to write announcement message in Markdown")
\nwused{\\{NW3wjrKS-2c3aOu-1}}\nwendcode{}\nwbegindocs{4}\nwdocspar


\section{The announce command function}

The main function that processes the announce command.
\nwenddocs{}\nwbegincode{5}\sublabel{NW3wjrKS-nRuDO-1}\nwmargintag{{\nwtagstyle{}\subpageref{NW3wjrKS-nRuDO-1}}}\moddef{functions~{\nwtagstyle{}\subpageref{NW3wjrKS-nRuDO-1}}}\endmoddef\nwstartdeflinemarkup\nwusesondefline{\\{NW3wjrKS-2c3aOu-1}}\nwprevnextdefs{\relax}{NW3wjrKS-nRuDO-2}\nwenddeflinemarkup
def announce_command(config, canvas, args):
  """Posts announcements to matching courses"""
  
  # Get list of matching courses
  course_list = list(canvaslms.cli.courses.filter_courses(canvas, args.course))
  
  if not course_list:
    print(f"No courses found matching pattern: \{args.course\}", file=sys.stderr)
    sys.exit(1)
  
  # Get the message content
  if args.interactive:
    message = get_message_from_editor()
  elif args.message:
    message = args.message
  else:
    print("Error: Either provide -m/--message or use -i/--interactive mode", file=sys.stderr)
    sys.exit(1)
  
  if not message.strip():
    print("Error: Message cannot be empty", file=sys.stderr)
    sys.exit(1)
  
  # Show courses that will receive the announcement
  print(f"Will post announcement '\{args.title\}' to \{len(course_list)\} course(s):")
  for course in course_list:
    print(f"  - \{course.course_code\}: \{course.name\}")
  
  # Confirm before posting
  confirm = input("Continue? (y/N): ")
  if confirm.lower() not in ['y', 'yes']:
    print("Cancelled.")
    sys.exit(0)
  
  # Post announcements
  success_count = 0
  for course in course_list:
    try:
      \LA{}post announcement to course~{\nwtagstyle{}\subpageref{NW3wjrKS-26JV7M-1}}\RA{}
      success_count += 1
      print(f"✓ Posted to \{course.course_code\}: \{course.name\}")
    except Exception as e:
      print(f"✗ Failed to post to \{course.course_code\}: \{course.name\} - \{str(e)\}", file=sys.stderr)
  
  print(f"Successfully posted to \{success_count\}/\{len(course_list)\} courses.")
\nwalsodefined{\\{NW3wjrKS-nRuDO-2}}\nwused{\\{NW3wjrKS-2c3aOu-1}}\nwendcode{}\nwbegindocs{6}\nwdocspar


\section{Interactive message editing}

Function to open an editor for writing the announcement message.
\nwenddocs{}\nwbegincode{7}\sublabel{NW3wjrKS-nRuDO-2}\nwmargintag{{\nwtagstyle{}\subpageref{NW3wjrKS-nRuDO-2}}}\moddef{functions~{\nwtagstyle{}\subpageref{NW3wjrKS-nRuDO-1}}}\plusendmoddef\nwstartdeflinemarkup\nwusesondefline{\\{NW3wjrKS-2c3aOu-1}}\nwprevnextdefs{NW3wjrKS-nRuDO-1}{\relax}\nwenddeflinemarkup
def get_message_from_editor():
  """Opens the user's preferred editor to write the announcement message"""
  editor = os.environ.get('EDITOR', 'nano')
  
  with tempfile.NamedTemporaryFile(mode='w+', suffix='.md', delete=False) as temp_file:
    temp_file.write("# Write your announcement message here\\n")
    temp_file.write("# Lines starting with # are comments and will be removed\\n")
    temp_file.write("# You can use Markdown formatting\\n\\n")
    temp_file_path = temp_file.name
  
  try:
    # Open editor
    subprocess.run([editor, temp_file_path], check=True)
    
    # Read the content back
    with open(temp_file_path, 'r') as temp_file:
      content = temp_file.read()
    
    # Remove comment lines and strip whitespace
    lines = []
    for line in content.split('\\n'):
      if not line.strip().startswith('#'):
        lines.append(line)
    
    return '\\n'.join(lines).strip()
    
  finally:
    # Clean up temporary file
    try:
      os.unlink(temp_file_path)
    except OSError:
      pass
\nwused{\\{NW3wjrKS-2c3aOu-1}}\nwendcode{}\nwbegindocs{8}\nwdocspar


\section{Posting announcements}

Function to post an announcement to a specific course.
\nwenddocs{}\nwbegincode{9}\sublabel{NW3wjrKS-26JV7M-1}\nwmargintag{{\nwtagstyle{}\subpageref{NW3wjrKS-26JV7M-1}}}\moddef{post announcement to course~{\nwtagstyle{}\subpageref{NW3wjrKS-26JV7M-1}}}\endmoddef\nwstartdeflinemarkup\nwusesondefline{\\{NW3wjrKS-nRuDO-1}}\nwenddeflinemarkup
discussion_topic = course.create_discussion_topic(
  title=args.title,
  message=message,
  is_announcement=True,
  published=True
)
\nwused{\\{NW3wjrKS-nRuDO-1}}\nwendcode{}

\nwixlogsorted{c}{{add arguments}{NW3wjrKS-Z37aL-1}{\nwixu{NW3wjrKS-2c3aOu-1}\nwixd{NW3wjrKS-Z37aL-1}}}%
\nwixlogsorted{c}{{announce.py}{NW3wjrKS-2c3aOu-1}{\nwixd{NW3wjrKS-2c3aOu-1}}}%
\nwixlogsorted{c}{{functions}{NW3wjrKS-nRuDO-1}{\nwixu{NW3wjrKS-2c3aOu-1}\nwixd{NW3wjrKS-nRuDO-1}\nwixd{NW3wjrKS-nRuDO-2}}}%
\nwixlogsorted{c}{{post announcement to course}{NW3wjrKS-26JV7M-1}{\nwixu{NW3wjrKS-nRuDO-1}\nwixd{NW3wjrKS-26JV7M-1}}}%
\nwbegindocs{10}\nwdocspar
\nwenddocs{}
